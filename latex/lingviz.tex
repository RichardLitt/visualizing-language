\documentclass[11pt]{article}
\usepackage{eacl2012}
\usepackage{times}
\usepackage{latexsym}
\usepackage{amsmath}
\usepackage{tikz}
\usepackage{graphicx}
\usepackage{multirow}
\usepackage{url}
\DeclareMathOperator*{\argmax}{arg\,max}
\setlength\titlebox{6.5cm}    % Expanding the titlebox

\title{Visualising Typological Relationships: Plotting WALS with Heat Maps}

 \author{Richard Littauer \\
 University of Saarland\\
 Computational Linguistics Department\\
 Saarbr\"ucken, Germany\\
   {\tt richard.littauer@gmail.com} \\\And
 Rory Turnbull \\
 Ohio State University\\
 Department of Linguistics\\
 Columbus, Ohio\\
   {\tt turnbull@ling.osu.edu} \\\AND
 Alexis Palmer\\
 University of Saarland\\
 Computational Linguistics Department\\
 Saarbr\"ucken, Germany\\
   {\tt apalmer@coli.uni-sb.de}\\} 

\date{}

\begin{document}
\maketitle

\begin{abstract}
This paper presents a novel way of visualising relationships between languages. The key innovation of the visualisation is that it brings geographic, phylogenetic, and linguistic data together into a single image, allowing a new visual perspective on linguistic typology. The data presented here is extracted from the World Atlas of Language Structures (WALS)~\cite{wals-2011}. After pruning due to low coverage of WALS, we filter data about linguistic structures and attributes by geographical proximity in order to get at areal typological effects. The data are displayed in heat maps which reflect the strength of similarity between languages for different linguistic aspects. Finally, these are annotated for language family membership. The images so produced allow a new perspective on the data which we hope may facilitate interesting findings and perhaps even illuminate new areas of research.
\end{abstract}

\section{Introduction}

This paper presents a novel way of visualising relationships between languages. Relationships between languages can be understood with respect to linguistic features of the languages, their geographical proximity, and their status with respect to historical development. The visualisations presented in this paper are the first to bring together these three perspectives into a single image. One line of recent work brings computational methods to bear on the formation and use of large typological databases, often using sophisticated statistical techniques to discover relations between languages \cite[among others]{cysouw2011,daume07implication,daume09areal}, and another line of work uses typological data in natural language processing \cite[for example]{georgi:etal:10,lewis:xia:08}, but we are unaware of any previous approaches to visually presenting the resulting data. Here, we address this gap by using data from the World Atlas of Language Structures \cite{wals-2011} to develop heat maps that can visually show the interconnected relationships between languages and language families. 

The main envisioned application of our visualisations is in the area of linguistic typology. Typology has been used to derive implications about possible languages, and about the ordering of the human mind. Different theorists have taken different views on the relationship between typology and the universality of languages. For example, Greenberg \shortcite{greenberg}, a foundational work, identifies a number of cross-linguistic typological properties and implications and aims to present them as truly universal -- relevant for \textit{all} languages. In a similar vein, typological universals have been employed as evidence in a generative story regarding language learning \cite{chomsky}.

Taking a different perspective, Dunn {\it et al} \shortcite{Dunn_Greenhill_Levinson_Gray_2011} argued that a language's typology relies upon the previous generations' language more than on any biological, environmental or cognitive constraints, and that there are pathways which are generally followed in language change based on the previous parent language. What these arguments have in common is a reliance on a view of linguistic typology that is potentially restricted in its scope, due to insufficient access to broad-scale empirical data, covering many features of many languages of the world. 

The most comprehensive computational resource for linguistic typology currently available is the World Atlas of Language Structures (WALS).\footnote{As of 2008, WALS is browsable online (\url{http://www.wals.info}).}  WALS is a large database of details of structural properties of several thousand languages \cite{wals-2011}. The properties were collected from descriptive sources by the project's 55 authors.

However, of the 2,678 languages and 192 features, only 16\% of the possible data points are actually specified---the data are \emph{sparse}, and the sparsity of the data of course makes it difficult to perform reliable statistical analysis. One way to work around this limitation is to seek meaningful visualisations of the data in WALS, instead of simply relying on raw numbers. This is our approach. 

In this paper, we first discuss in more detail the source data and the types of information to be extracted, followed by a discussion of some difficulties presented by the available data and our approaches for addressing those difficulties. Finally, we show the resulting visualisations.

\section{Aspects of the Visualisations}

The visualisations described here bring together three types of information: linguistic features, geographical distance, and phylogenetic distance. For the current study, all three types of information are extracted from the WALS database. In future work, we would explore alternate sources such as Ethnologue \cite{ethnologue} or MultiTree \shortcite{multitree} for alternate phylogenetic hierarchies. 

\subsection{Linguistic features}
At the time of submission, WALS contained information for 2,678 languages. The linguistic features covered in WALS range from phonetic and phonological features, over some lexical and morphological features, to syntactic structures, word order tendencies, and other structural questions. A total of 192 features are represented, grouped in 144 different chapters, with each chapter addressing a set of related features.  Ignoring the fact that a language having certain features will cancel out the possibility or probability of others, only 15.8\% of of WALS is described fully.

The coverage of features/chapters varies dramatically across the languages, with an average of 28 feature values per language. The most populated feature has data for 1,519 languages. Because of the extreme sparsity of the data, we restricted our treatment to only languages with values for 30\% or more of the available features. 


\subsection{Geographic distance}
Geographic distance is an important aspect of typological study because neighbouring  languages often come to share linguistic features, even in the absence of genetic relationship between the languages. Each language in WALS is associated with a geographical coordinate representing a central point for the main population of speakers of that language. We use these figures for determining geographic distance between two languages. A crucial aspect of our visualisations is that we produce them only for sets of languages within a reasonable geographic proximity (and with sufficient feature coverage within WALS).

For this study, we try two approaches to clustering languages according to geographic distance. First, we choose an arbitrary radius in order to create a decision boundary for clustering neighbouring languages. For each language, we fix that language's location as the centroid of the cluster and look at all other languages within the given radius. We found that a radius of 500 kilometres proves to provide a sufficient number of examples even after cleaning low-coverage languages from the WALS data. 

The second approach selects an arbitrary lower bound for the languages in the general area. If a sufficient percentage of the total number of languages in the area remained after cleaning the WALS data, we took this as a useful area and did mapping for that area. This number is clearly under-representative of the amount of contact languages, as only half of the world's languages are present in WALS. This proxy was not as good at choosing specific, useful examples, as the $n$-nearest neighbours, as the languages chosen were often too far away. 

\subsection{Phylogenetic distance}

Languages are related phylogenetically either vertically, by lineage, or horizontally, by contact. In WALS, each language is placed in a tree hierarchy that specifies phylogenetic relations. In the WALS data files, this is specified by linking at three different levels: family, such as `Sino-Tibetan', sub-family, such as `Tibeto-Burman', and genus, such as `Northern Naga' . The WALS phylogenetic hierarchies do not take into account language contact. For that, we used geographic coordinates, which are present on WALS, as a proxy for contact. 


\section{Heat Map Visualisations}

We focus our visualisations on languages with a reasonable number of filled features, as drawing a heat map where only 16\%~of the features are available would be of little use. There are two options for dealing with this: to collapse the feature values in some way, or to select for languages that have a higher percentage of data filled than the average language. We opted for the second choice, and \emph{cleaned}  the file until it contained only languages that had at least 30\% as a lower bound of all of their entries filled. This cleaned data was then used in the other functions. We further focused on producing visualisations only for features that are salient for the maximal number of selected languages. We choose two heat maps for display here, but all visualisations (and all code) are available from a public repository, links to which are withheld only to retain anonymity. 


All data was downloaded freely from WALS, all coding was done in either Python or R. The code was not computationally expensive to run, and the programming languages and methods are quite accessible. 
%All code is available in a public repository, links to which will be released once anonymisation is no longer required.

In a two-dimensional heat map, each cell of a matrix is filled with a colour representing that cell's value. In our case, the colour of the cell represents the normalised value of a linguistic feature according to WALS. Languages with the same colour in a given row have the same value for that typological feature.\footnote{Due to this reliance on colour, we strongly suggest viewing the heat maps presented here in colour.} Below we discuss two types of heat maps, focusing on either geographic or phylogenetic features.

\subsection{Geographically-focused heat maps}
For the geographic distance maps, for each language present in the cleaned data, we selected all possible languages that lay within 500km, and sorted these languages until only the 15 closest neighbours were selected. We picked features to graph from among the resulting languages based on how common they were across the selected languages. 

Each final list was then resorted. In the geographic case, the source language was centred in the map. This was due to one of the primary issues with using distance on a two dimensional graph. On the one hand, we would want close languages to be close together on the heat map. However, given the source language Egyptian Arabic, this would mean that Zulu, Saami, Lakota, and Japanese might all roughly be placed next to each other on the map. This also means that Japanese might be placed next to Lakota, and Lakota again next to Korean. This is not ideal, and was the main justification for limiting the sphere of possible geographical languages to a reasonable distance, given the data.


\begin{figure}[!b]
\includegraphics[width=3.1in]
{graph2yimassmall.pdf} 
\caption{Geographically-focused heat map; see text for details. The bar at the top of the image represents the language family of the language in that column: Pink = Border; Red = Trans-New Guinea; Blue = Sepik; Brown = Lower Sepik-Ramu; Purple = Torricelli; Green = Skou; and Orange = Sentani.} 
\label{fig:heat1} 

\end{figure}

Figure~\ref{fig:heat1} shows a geographically-focused heat map centred on Yimas, spoken in New Guinea, with various syntactic features. The features were chosen based on their relative absence of missing data. For complex constructions, the heat map shows partial grouping of languages closer to Yimas, and less similarity at a greater distance. The checkerboard pattern for dominant word orders may suggest groups that have been split by the data-centring function. In general, however, this graph shows that, for these features, the languages are for the most part homogenous. This is unlikely to be a chance effect, given the similarity in language families, as can be seen in the very top bar of the graph.

\subsection{Phylogenetically-focused heat maps}

For each language we searched for other languages coming from the same family, subfamily, or genus. Figure~\ref{fig:heat2}, shows a phylogenetically-focused heat map for Niger-Congo languages, arranged from east to west. It shows clear clusterings in the eastern languages, especially for negation orders. It also shows pronominal subject expression and agreement marking agreement for the western languages clearly. We also see some groupings of feature values in adjacent languages, for example: Bambara and Supyire. Especially given the importance of Bambara for syntactic argumentation\cite{culy}, this graph is an excellent example of visualisation pointing out an intriguing area for closer analysis.

\begin{figure}[ht!]
\includegraphics[width=3.1in]
{graph3nigercongosmall.pdf} 
\caption{Phylogenetic heat-map of Niger-Congo languages, arranged from east to west.} 
\label{fig:heat2} 
\end{figure}


\subsection{Sparse data}

\begin{figure}[h]
\includegraphics[width=3.1in]
{graph1sparsenew.pdf} 
\caption{Scatterplot of language location (x and y axes), family (colour), and density of feature specification in WALS (size).} 
\label{fig:sparse} 
\end{figure}

In Fig.~\ref{fig:sparse}, we have plotted out the neighbouring languages for the most populous language area represented on WALS by longitude and latitude. The centre language, Yimas, is a Trans-New Guinean language, and is the centre for the graph. The odd clustering is due to the Pacific Ocean--- such topographical features limit the usefulness of distance as a measure.  Another limit is the range of each language, which is not represented here, as each language is given only a single geographical coordinate that does not indicate the area over which the language resides, nor how many other languages coexist with it.

For each language within 500km the number of features specified in WALS is shown. Only a small minority have more than 50\% of their features filled---there are only 171 such languages in WALS. This graph does not include languages not in WALS. 


\section{Conclusion}
In this paper we present a new method for visualising relationships between languages, one which allows for the simultaneous viewing of linguistic features together with phylogenetic relationships and geographical location and proximity. These visualisations allow us to view relationships in a new way, seeking to work around the sparseness of available data and facilitate new insights into linguistic typology.

In this work we placed strong restrictions on both feature coverage and selection of salient features for representation, reducing the number of graphs produced to 6 with geographic focus and 8 with phylogenetic focus. One topic for future work is to explore other ways of working with and expanding the available data in order to access even more useful visualisations.

\newpage
\bibliographystyle{acl}
\bibliography{lingviz}

\end{document}
